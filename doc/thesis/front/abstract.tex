\chapter{Abstract}


This thesis is elaborated as the final project in the master's degree program Embedded Systems Design, at the University of Applied Sciences, Campus Hagenberg.
A comprehensive practical project is developed accompanying to the theoretical topics, which are elaborated in this thesis.\\

The objective of this thesis is the development of a system design for an embedded system.
A system-level design approach is taken to develop an FM radio receiver in multiple implementation variants, targeting a system-on-chip hardware, including an FPGA and a CPU.\\

Specifically, high-level synthesis and manual VHDL implementation is used, to develop two versions of the FPGA design.
This includes the digital signal processing chain, as well as multiple interfaces to communicate with the design.
The two variants are compared, based on specific metrics, such as implementation time, lines of code, hardware utilization, and the general usability experience during the development process.
At the end, a productive system is deployed on actual hardware, including a firmware application to provide a basic user interface.\\

The entire development process is determined by the general point of view of system design.
Therefore, a system model is developed in the first stage, by using Matlab, which allows the elaboration of algorithms for digital signal processing and the creation of a system, which can be implemented in hardware.
Furthermore, GNU Radio is used to establish a prototype, which can be deployed on actual hardware.
The entire system, with its two main implementation methods and the Matlab system model, is using a common verification environment, which guarantees a correctly functioning design.
This verification is performed in simulation and on the actual hardware, which closes the loop of the system design approach.
