%%%%%%%%%%%%%%%%%%%%
\chapter{Comparison and Results}
\label{cha:ComparisonAndResults}
  \section{Signal Output}
  % show plots of the common analyzer python script

  \section{Functionality}
  % Functionality same?
  % DSP yes, AXI no (FSM of axi-stream)


  \section{Code Development}
  % lines of code
  % Implementation effort/time

  \section{Simulation}
  % Sim speed of x min of recorded FM signal

  \section{Useability}
  % IP generation and integration into Vivado

  \section{Hardware Utilization}
  % especially VHDL<->HLS

  \section{Latency}
  % especially VHDL<->HLS

  % NOTES:
  % - The system level architecture that was chosen, can be applied to any DSP project. Simply replace the main IP, e.g. from FM to DAB+, QAM, etc. and the environment can run in the exact same way.
  %
  % Further steps:
  % - generate VHDL from Matlab receiver code
  % - use the common input file and feed it to GNU Radio; also produce an output file with GNU Radio for comparison
  %- Develop HLS compiler independent code (optional)}
  %   -- Different compilers use different #pragmas, etc.
  %   -- These #pragmas need to be within the code, at the appropriate positions, which makes it difficult to write compiler independent code....
