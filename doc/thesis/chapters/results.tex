%%%%%%%%%%%%%%%%%%%%
\chapter{Comparison and Results}
\label{cha:ComparisonAndResults}

This chapter presents the results that are achieved in the development of the FM Radio Receiver project, with the chosen different methods.
However, the main focus is to compare the results that are achieved with HLS and VHDL, since these are the major implementation variants of the project.

%%%%%%%%%%%%%%%%%%%%%%%%%%%%%%%%%%%%%%%%%%%%%%%%%%%%%%%%%%%%%%%%%%%%%%%%%%%%%%
\section{General}

A main objective of this thesis is to implement an FM radio receiver in multiple different methods.
This includes GNU Radio, Matlab, VHDL and HLS.
The target functionality is to listen to the audio broadcast signal of an FM radio station.\\

All of the listed methods are implemented and the target functionality is achieved with all of them.
However, the level of detail that is required in the implementation, and the resulting effort and time it takes to implement the respective variant, differs by a large factor.
This is mainly due to the different levels of abstraction, so that the low-level algorithms do not need to be known.\\

In the following sections, especially the HLS and VHDL implementations are compared on the basis of metrics, but also based on the experience during development.


%%%%%%%%%%%%%%%%%%%%%%%%%%%%%%%%%%%%%%%%%%%%%%%%%%%%%%%%%%%%%%%%%%%%%%%%%%%%%%
\section{Functionality}

Generally, both tools --- HLS and VHDL --- provide the capabilities to implement any functionality in one or the other method.
However, the implementation is done in a different level.
VHDL uses an approach that is very close to the hardware, such as clock cycles and flipflops, while HLS describes the logic on an algorithmic level.\\

The implementation is split into two main parts, the communication interfaces and the DSP.
The results are presented in the following sections.

% Functionality same?
% DSP yes, AXI no (FSM of axi-stream)
% unit tests

%%%%%%%%%%%%%%%%%%%%
\subsection{Interfaces}


%%%%%%%%%%%%%%%%%%%%
\subsection{Audio Output}

Regarding the DSP, both implementation variants achieve a very similar result.

% show plots of the common analyzer python script
Demodulated Signal

\includepicture [1.0] [0] {Comparison of the audio signal.} {audio_output_compare_all_ips} {img/matlab/audio_output_compare_all_ips}


%%%%%%%%%%%%%%%%%%%%%%%%%%%%%%%%%%%%%%%%%%%%%%%%%%%%%%%%%%%%%%%%%%%%%%%%%%%%%%
\section{Code Development}

%%%%%%%%%%%%%%%%%%%%
\subsection{Lines of Code}
% of the entire project

%%%%%%%%%%%%%%%%%%%%
\subsection{Implementation Time}

% Implementation effort/time
%    -- see logged times; is subjective; had no previous knowledge
%    -- AXI interfaces --> huge effort in VHDL...
%        (mention Johannes Walter for generously providing/allowing the Register Engine script, adapted for this thesis)
%    -- drivers are auto-generated with HLS! huge time-saving..
%       (maybe show the code of AXI4-Lite interface and its generated driver
%        --> difference between R/W and RO!;
%        and show screenshots of the IP in the block design)

%%%%%%%%%%%%%%%%%%%%%%%%%%%%%%%%%%%%%%%%%%%%%%%%%%%%%%%%%%%%%%%%%%%%%%%%%%%%%%
\section{Simulation}
% Sim speed of x min of recorded FM signal


%%%%%%%%%%%%%%%%%%%%%%%%%%%%%%%%%%%%%%%%%%%%%%%%%%%%%%%%%%%%%%%%%%%%%%%%%%%%%%
\section{System Design}

% - The system level architecture that was chosen, can be applied to any DSP project. Simply replace the main IP, e.g. from FM to DAB+, QAM, etc. and the environment can run in the exact same way.


%%%%%%%%%%%%%%%%%%%%%%%%%%%%%%%%%%%%%%%%%%%%%%%%%%%%%%%%%%%%%%%%%%%%%%%%%%%%%%
\section{Deployment on Hardware}

%%%%%%%%%%%%%%%%%%%%
\subsection{Integration}
% IP generation and integration into Vivado

%%%%%%%%%%%%%%%%%%%%
\subsection{Hardware Utilization}
% --> utilization reports

%%%%%%%%%%%%%%%%%%%%
\subsection{Latency}
% --> utilization reports

%%%%%%%%%%%%%%%%%%%%%%%%%%%%%%%%%%%%%%%%%%%%%%%%%%%%%%%%%%%%%%%%%%%%%%%%%%%%%%
\section{Applicability}
% of all implementation variants
%   GNU Radio for fast prototyping,
%   Matlab for research
%   HLS/VHDL for deployment on HW, product development

