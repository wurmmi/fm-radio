%%%%%%%%%%%%%%%%%%%%
\chapter{Comparison and Results}
\label{cha:ComparisonAndResults}

This chapter presents the results that are achieved in the development of the FM Radio Receiver project, with the chosen different methods.
However, the main focus is to compare the results that are achieved with HLS and VHDL, since these are the major implementation variants of the project.

%%%%%%%%%%%%%%%%%%%%%%%%%%%%%%%%%%%%%%%%%%%%%%%%%%%%%%%%%%%%%%%%%%%%%%%%%%%%%%
\section{General}

A main objective of this thesis is to implement an FM radio receiver in multiple different methods.
This includes GNU Radio, Matlab, VHDL and HLS.
The target functionality is to listen to the audio broadcast signal of an FM radio station.\\

All of the listed methods are implemented and the target functionality is achieved with all of them.
However, the level of detail that is required in the implementation, and the resulting effort and time it takes to implement the respective variant, differs by a large factor.
This is mainly due to the different levels of abstraction, so that the low-level algorithms do not need to be known.\\

In the following sections, especially the HLS and VHDL implementations are compared on the basis of metrics, but also based on the experience during development.


%%%%%%%%%%%%%%%%%%%%%%%%%%%%%%%%%%%%%%%%%%%%%%%%%%%%%%%%%%%%%%%%%%%%%%%%%%%%%%
\section{Functionality}

Generally, both tools -- HLS and VHDL -- provide the capabilities to implement any functionality in one or the other method.
However, the implementation is done in a different level.
VHDL uses an approach that is very close to the hardware, such as clock cycles and flipflops, while HLS describes the logic on an algorithmic level.\\

The implementation is split into two main parts, the communication interfaces and the DSP.
The results are presented in the following sections.

% Functionality same?
% DSP yes, AXI no (FSM of axi-stream)
% unit tests

%%%%%%%%%%%%%%%%%%%%
\subsection{Interfaces}

The AXI4-Lite memory-mapped bus is implemented to have an exactly equal behaviour in both variants.
There are read-only and read-writeable registers, which are all mapped onto a single base address.\\

The AXI stream interface however does not show the exact same behaviour.
Here, the HLS variant is using the AXI stream according to its protocol, while the VHDL variant is implemented differently.
It uses a simplified logic for the ready-flag, which reduces the effort in implementation.
However, from the perspective of the communicating blocks, the interface is usable like a regular AXI stream.
The implementation details are explained in Section \ref{sec:impl:vhdl:interfaces}.\\

In the final, integrated system on the SoC hardware, the CPU is able to communicate with both IPs via their interfaces successfully.
Status and configuration data can be read and written through the AXI4-Lite interface, and the streaming data for the DSP chain is successfully sent through the AXI streams as well.

%%%%%%%%%%%%%%%%%%%%
\subsection{Audio Output}

The audio output of the HLS and VHDL variants is compared with the Matlab model, which serves as the reference.
Additionally, the results of the testbench are compared with their respective result in the actual hardware.
In summary, it can be stated that the DSP chain produces a very similar audio output in all the compared data sets.
However, differences remain, which are presented in the following paragraphs and diagrams.

\includepicture [1.0] [0] {Comparison of the IP audio output signals, in simulation and hardware, against the Matlab model. The HLS variant matches the model, while the VHDL output diverts by a certain amount. Also, VHDL differs between simulation- and hardware results.} {audio_output_compare_tb_vs_hw} {img/matlab/audio_output_compare_tb_vs_hw}

%%%%%
\subsubsection{Comparing against the Matlab reference}

In the comparison against the Matlab model, the HLS variant achieves a very exact match of the output signal.
However, the VHDL implementation differs by a certain amount.
There seems to be an issue in the signal separation between the left and right audio channels.
This is visible in the analysis of the audio signal, as shown is Fig. \ref{fig:audio_output_compare_tb_vs_hw}.
The signal strength of the left channel in the upper diagram is weaker than expected, while the right channel in the lower diagram is stronger.
This imbalance can also be observed by listening to the audio output via speakers.\\

The cause for this issue may be located in several components in the design.
This includes the FM demodulator, the carrier recovery, i.e. the 38~kHz carrier, as well as a sample timing shift in the final summation and substraction to recover the left and right channels.
The FIR filter has a successfully passing unit test and is thus assumed to be correct.
Also the fixed-point data type with its overflow- and rounding behaviour is suspected to be a potential issue.
Due to the time limitation in the elaboration of this thesis, this issue can not be traced down to the root cause and thus still persists in the current implementation.
However, from the system-level point of view of this thesis, the IP works and can be integrated into the final system.

%\includepicture [1.0] [0] {Comparison of the IP audio output signals against the Matlab model. The HLS variant matches the model, while the VHDL output diverts by a certain amount.} {audio_output_compare_ips_vs_matlab} {img/matlab/audio_output_compare_ips_vs_matlab}

%%%%%
\subsubsection{Comparing simulation- against hardware results}

Here, the results of the simulation testbenches are compared against the values that are read from the FPGA directly.
Ideally, the values should be exactly the same.
However, Figure \ref{fig:audio_output_compare_tb_vs_hw} shows that this is not always the case.
Again, the HLS implementation does have a matching result, whereas the VHDL variant has deviating values.
The explanation here may be linked with above suspicions, but may also be linked to the initial reset values that exist in the hardware.
In the HLS implementation, the reset logic of several registers is implemented automatically, whereas in VHDL these conditions need to be implemented manually.
Therefore, the IPs internal state, i.e. the register values, may be different at the beginning, which leads to an error propagation throughout the design.
The entire DSP chain is built like a pipeline and therefor it takes a number of samples to process, before the 'wrong' intermediate values are flushed out.
All that may be the cause for the deviating values in the VHDL design.\\

Again, the time limitation in the elaboration of this thesis does not allow a deeper analysis of this issue.
However, the end-to-end system design enables the developer to analyze an issue like this in a convenient way, in quick iterations.
Any adaptions in the VHDL design can be simulated and integrated into the FPGA with automated scripts.
Both results can then be analyzed and compared, which may lead to further adaptions.\\


%%%%%%%%%%%%%%%%%%%%%%%%%%%%%%%%%%%%%%%%%%%%%%%%%%%%%%%%%%%%%%%%%%%%%%%%%%%%%%
\section{Code Development}

asdf

%%%%%%%%%%%%%%%%%%%%
\subsection{Lines of Code}

% of the entire project
asdf

%%%%%%%%%%%%%%%%%%%%
\subsection{Implementation Time}

% Implementation effort/time
%    -- see logged times; is subjective; had no previous knowledge
%    -- AXI interfaces --> huge effort in VHDL...
%        (mention Johannes Walter for generously providing/allowing the Register Engine script, adapted for this thesis)
%    -- drivers are auto-generated with HLS! huge time-saving..
%       (maybe show the code of AXI4-Lite interface and its generated driver
%        --> difference between R/W and RO!;
%        and show screenshots of the IP in the block design)

%%%%%%%%%%%%%%%%%%%%%%%%%%%%%%%%%%%%%%%%%%%%%%%%%%%%%%%%%%%%%%%%%%%%%%%%%%%%%%
\section{Simulation}
% Sim speed of x min of recorded FM signal


%%%%%%%%%%%%%%%%%%%%%%%%%%%%%%%%%%%%%%%%%%%%%%%%%%%%%%%%%%%%%%%%%%%%%%%%%%%%%%
\section{System Design}

% - The system level architecture that was chosen, can be applied to any DSP project. Simply replace the main IP, e.g. from FM to DAB+, QAM, etc. and the environment can run in the exact same way.


%%%%%%%%%%%%%%%%%%%%%%%%%%%%%%%%%%%%%%%%%%%%%%%%%%%%%%%%%%%%%%%%%%%%%%%%%%%%%%
\section{Deployment on Hardware}

%%%%%%%%%%%%%%%%%%%%
\subsection{Integration}
% IP generation and integration into Vivado

%%%%%%%%%%%%%%%%%%%%
\subsection{Hardware Utilization}
% --> utilization reports

%%%%%%%%%%%%%%%%%%%%
\subsection{Latency}
% --> utilization reports

%%%%%%%%%%%%%%%%%%%%%%%%%%%%%%%%%%%%%%%%%%%%%%%%%%%%%%%%%%%%%%%%%%%%%%%%%%%%%%
\section{Applicability}
% of all implementation variants
%   GNU Radio for fast prototyping,
%   Matlab for research
%   HLS/VHDL for deployment on HW, product development

