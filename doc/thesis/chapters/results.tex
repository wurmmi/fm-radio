%%%%%%%%%%%%%%%%%%%%
\chapter{Comparison and Results}
\label{cha:ComparisonAndResults}

% especially VHDL<->HLS

%%%%%%%%%%%%%%%%%%%%%%%%%%%%%%%%%%%%%%%%%%%%%%%%%%%%%%%%%%%%%%%%%%%%%%%%%%%%%%
\section{Demodulated Signal}
% show plots of the common analyzer python script

%%%%%%%%%%%%%%%%%%%%%%%%%%%%%%%%%%%%%%%%%%%%%%%%%%%%%%%%%%%%%%%%%%%%%%%%%%%%%%
\section{Functionality}
% Functionality same?
% DSP yes, AXI no (FSM of axi-stream)
% unit tests

%%%%%%%%%%%%%%%%%%%%%%%%%%%%%%%%%%%%%%%%%%%%%%%%%%%%%%%%%%%%%%%%%%%%%%%%%%%%%%
\section{Code Development}
% lines of code
% Implementation effort/time
%    -- see logged times; is subjective; had no previous knowledge
%    -- AXI interfaces --> huge effort in VHDL...
%        (mention Johannes Walter for generously providing/allowing the Register Engine script, adapted for this thesis)
%    -- drivers are auto-generated with HLS! huge time-saving..
%       (maybe show the code of AXI4-Lite interface and its generated driver
%        --> difference between R/W and RO!;
%        and show screenshots of the IP in the block design)

%%%%%%%%%%%%%%%%%%%%%%%%%%%%%%%%%%%%%%%%%%%%%%%%%%%%%%%%%%%%%%%%%%%%%%%%%%%%%%
\section{Simulation}
% Sim speed of x min of recorded FM signal

%%%%%%%%%%%%%%%%%%%%%%%%%%%%%%%%%%%%%%%%%%%%%%%%%%%%%%%%%%%%%%%%%%%%%%%%%%%%%%
\section{Useability}
% IP generation and integration into Vivado

%%%%%%%%%%%%%%%%%%%%%%%%%%%%%%%%%%%%%%%%%%%%%%%%%%%%%%%%%%%%%%%%%%%%%%%%%%%%%%
\section{Applicability}
% of implementation variant (GNU Radio for fast prototyping, Matlab for research, HLS/VHDL deployment on HW, etc.)

%%%%%%%%%%%%%%%%%%%%%%%%%%%%%%%%%%%%%%%%%%%%%%%%%%%%%%%%%%%%%%%%%%%%%%%%%%%%%%
\section{Advantages / Disadvantages}

easy and much faster to get complex HDL code, such as image/video processing, through libraries like OpenCV.\\
However, it may be less optimized.\\
A developer still needs to have a strong hardware background, to be able to understand what is being generated in hardware from x lines of code in HLS/Cpp.

%%%%%%%%%%%%%%%%%%%%%%%%%%%%%%%%%%%%%%%%%%%%%%%%%%%%%%%%%%%%%%%%%%%%%%%%%%%%%%
\section{Hardware Utilization}
% --> utilization reports

%%%%%%%%%%%%%%%%%%%%%%%%%%%%%%%%%%%%%%%%%%%%%%%%%%%%%%%%%%%%%%%%%%%%%%%%%%%%%%
\section{Latency}
% --> utilization reports

%%%%%%%%%%%%%%%%%%%%%%%%%%%%%%%%%%%%%%%%%%%%%%%%%%%%%%%%%%%%%%%%%%%%%%%%%%%%%%
  \section{Lines of Code}
  % of the entire project

  % NOTES:
  % - The system level architecture that was chosen, can be applied to any DSP project. Simply replace the main IP, e.g. from FM to DAB+, QAM, etc. and the environment can run in the exact same way.
  %
  % Further steps:
  % - generate VHDL from Matlab receiver code
  % - use the common input file and feed it to GNU Radio; also produce an output file with GNU Radio for comparison
  %- Develop HLS compiler independent code (optional)}
  %   -- Different compilers use different #pragmas, etc.
  %   -- These #pragmas need to be within the code, at the appropriate positions, which makes it difficult to write compiler independent code....
