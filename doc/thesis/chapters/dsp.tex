%%%%%%%%%%%%%%%%%%%%
\chapter{Signal Processing Theory}
\label{cha:TheThesis}

%%%%%%%%%%%%%%%%%%%%
\section{Modulation from Baseband to RF}

%%%%%%%%%%%%%%%%%%%%
\section{Modulation from RF to Baseband}

%%%%%%%%%%%%%%%%%%%%
\section{Frequency Modulation in Broadcasting}
  %see literature/Present\_lec6\_AM\_FM.pdf\\
  %see lectures out of HSD/ESD

Frequency modulation (FM) is a widely used standard to transmit data streams.
The probably best known usecase therefore is commercial broadcast radio, where an audio stream is transmitted.
Devices to receive these streams are available for low prices to the public.

This chaper describes the main properties of broadcast FM, such as the mathematical description, frequency bands that are used, or the specific frequency parts within a channel spectrum.

%%%%%%%%%%%%%%%%%%%%
\subsection{Mathematical description}

The mathematical description of an FM baseband signal can be described with the formulas presented in this section.
FM encodes the information content in its instantanious frequency.
This means that the measured frequency at any moment in time represents a specific value of a transmitted message.
For a general classification, FM belongs to the group of angle, or phase modulated signals (PM).
The simple reason therefor is, that a frequency has a direct relationship to an angle, if the signal is seen on a unit circle.\\

The instantanious frequency $f_i$ of an FM signal can be described as
\begin{equation}
  f_i = f_c + \Delta f \cdot m(t)
\end{equation}
where $f_c$ is the carrier- or center frequency, $\Delta f$ is the maximum frequency deviation and $m(t)$ is the information or message signal that is to be transmitted.
Simply looking at this equation, the frequency deviation, which is sometimes also called the swing, varies in the range between $f_c \pm \Delta f$.

Considering the relationship between frequency and angle, as described above, and after some simple substitutions which will not be described into detail here, the equation for a generic frequency modulated wave is the following
\begin{equation}
  s(t) = A_c\ cos \Big( 2 \pi f_c t + 2 \pi k_f \int m(t) dt \Big)
  \label{equ:fm_func}
\end{equation}
where $A_c$ is the amplitude of the resulting FM signal and $k_f$ is the frequency sensitivity factor.
This sensitivity factor has a direct relationship with the modulation index $\beta$.
\begin{equation}
  \beta = \frac{\Delta f}{f_m} = \frac{k_f A_m}{f_m}
\end{equation}
where $f_m$ is the highest existing frequency, or the bandwidth of the information signal.

Formula \ref{equ:fm_func} describes a frequency modulated signal with a generic message signal $m(t)$.
A widely used example application of FM is broadcast radio, where audio streams are transmitted.
An audio stream can be described as a cosine wave.
Therefor, the information signal that is to be transmitted in broadcast radio can be represented as a cosine wave in the form of
\begin{equation}
  m(t) = A_m\ cos(2 \pi f_m t).
\end{equation}
By inserting this message signal into the generic FM formula \ref{equ:fm_func}, the final equation for an FM signal transforms into the following form.
\begin{equation}
  y(t) = A_c\ cos \Big(2 \pi f_c t + \beta\ sin(2 \pi f_m t)\Big )
\end{equation}
Several formulas, as well as their derivations are described according to \cite{ref_FM_Maths_Info_1}\cite{ref_FMMaths2}.

%%%%%%%%%%%%%%%%%%%%
\subsection{Frequency Band, Channel allocation/distribution}

%%%%%%%%%%%%%%%%%%%%
\section{Algorithms for Digital FM Demodulation}
  see literature/FmDemodulator.pdf (Sect. 3.3)
  see literature/00476180 Digital FM Demodulator for FM, TV, and Wireless.pdf (Sect. II and III)

  %%%%%%%%%%%%%%%%%%%%
  \subsection{Baseband Delay Demodulator}
  %%%%%%%%%%%%%%%%%%%%
  \subsection{Phase-Adapter Demodulator}
  %%%%%%%%%%%%%%%%%%%%
  \subsection{Phase-Locked Loop (PLL)}
  %%%%%%%%%%%%%%%%%%%%
  \subsection{Mixed Demodulator}
