%%%%%%%%%%%%%%%%%%%%
\chapter{Conclusion and Prospect}
\label{cha:ConclusionAndProspect}

This thesis covers a comprehensive variety of topics, that ranges from digital signal processing to system modeling and -design, IP design in different methods, verification, to hardware-software interaction in hardware.
The implementation of the FM radio receiver in multiple different methods, namely GNU Radio, Matlab, VHDL and HLS so that they achieve the same functionality, is a challenge, especially with the latter methods.
However, this challenge is solved within an end-to-end system design, that strongly supports the development process.\\

GNU Radio is a software, that allows an efficient development of a software-defined radio as a prototype, which can run on actual hardware in a live environment.
It is definitely worth to use it during the development of a project like the one in this thesis, since it allows to achieve a functioning result with relatively low effort.\\

The usage of Matlab as a tool to research DSP algorithms and strategies to demodulate a signal, in order to ultimately find a system model that can be implemented in hardware, turns out to be a powerful asset.
It allows the development in fast iterations, through short simulation times, to achieve a result quickly.
Based on this system model, the implementation of actual hardware IPs, with VHDL and HLS can be done more efficiently and in a direct way, because issues in the algorithm level are already fixed in the model at this point.
In combination with the simulation testbenches that share major parts, such as the input source, as well as a the analysis tool, the IP development works smoothly and efficiently.
The verification as a comparison against the Matlab system model gives instant feedback of whether the IP functions correctly.
Even if the system model changes, the testbench automatically adapts, since the input source files are re-generated by the model and thus picked up at the next iteration of the testbench simulation, so the hardware engineer can also adapt the IP code.
Once the IP is deployed on actual hardware, it is tested by the firmware, in order to check its correct functionality again, which closes the loop from the system model and the IP simulation to the hardware deployment.
This entire process proved itself to be a successful approach.\\

Regarding the implementation of the VHDL and HLS IPs and the accompanying comparison, the result turns out to be unexpected.
The initial expectation, that HLS is inefficient in terms of resource utilization, is contradicted.
Also the amount of time spent in the implementation and the metric of lines of code create a positive picture of HLS.
Further, the automatic generation of several files that are required to integrate the IP into a system design, such as interface definitions or software drivers, support this attitude.
VHDL may have its advantages in specific areas of application, but HLS is definitely a powerful tool that can achieve great results in less time.
This is especially true, when it comes to the development of complex functionalities, such as image- or video processing, with the support of built-in libraries like OpenCV.\\

However, the often heard statement that HLS enables any software developer to develop hardware IPs, can not be supported.
A developer that writes HLS code, still needs to have a strong hardware background, to be able to understand what is being generated in hardware.
It is important to know about the limitations of the target hardware, i.e. FPGA, in order to write an efficient design.
Therefore, the connection between specific HLS \cplusplus\ code lines, as well as the general structure and algorithm, and the hardware target need to be understood.




% NOTES:
%
% Further steps:
%  - generate VHDL from Matlab receiver code
%  - use the common input file and feed it to GNU Radio;
%    also produce an output file with GNU Radio for comparison
%  - Develop HLS compiler independent code (optional)}
%    -- Different compilers use different #pragmas, etc.
%    -- These #pragmas need to be within the code, at the appropriate positions,
%       which makes it difficult to write compiler independent code....
%  - use Matlab Simulink HDL Coder and System Generator
%     as additional implementation variants
%  - use Intel oneAPI HLS compiler
%      -- try to develop compiler-independent code
%      -- how platform-independent is HLS?
%  - explore HLS optimizations
%  - explore HLS libraries (e.g. DSP library --> pre-made FIR filter)
%  - create another IP in the same system
%      -- due to the modular system design this is easy to do
%  - use Linux OS on the SoC, to enable live-data streaming with the RTL-SDR
%
%
%
% ...the list shows how comprehensive the topic of this thesis is.
% Based on the results that were made during the elaboration, many more ideas came into play.
% Many ideas to extend the project, ...
