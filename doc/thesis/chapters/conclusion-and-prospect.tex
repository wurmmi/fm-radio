%%%%%%%%%%%%%%%%%%%%
\chapter{Conclusion and Prospect}
\label{cha:ConclusionAndProspect}

%\section{Applicability}
% of all implementation variants
%   GNU Radio for fast prototyping,
%   Matlab for research
%   HLS/VHDL for deployment on HW, product development


% NOTES:
%
% Further steps:
%  - generate VHDL from Matlab receiver code
%  - use the common input file and feed it to GNU Radio;
%    also produce an output file with GNU Radio for comparison
%  - Develop HLS compiler independent code (optional)}
%    -- Different compilers use different #pragmas, etc.
%    -- These #pragmas need to be within the code, at the appropriate positions,
%       which makes it difficult to write compiler independent code....
%  - use Matlab Simulink HDL Coder and System Generator
%     as additional implementation variants
%  - use Intel oneAPI HLS compiler
%      -- try to develop compiler-independent code
%      -- how platform-independent is HLS?
%  - explore HLS optimizations
%  - explore HLS libraries (e.g. DSP library --> pre-made FIR filter)
%  - create another IP in the same system
%      -- due to the modular system design this is easy to do
%  - use Linux OS on the SoC, to enable live-data streaming with the RTL-SDR
%
%
%
% ...the list shows how comprehensive the topic of this thesis is.
% Based on the results that were made during the elaboration, many more ideas came into play.
% Many ideas to extend the project, ...

%%%%%%%%%%%%%%%%%%%%%%%%%%%%%%%%%%%%%%%%%%%%%%%%%%%%%%%%%%%%%%%%%%%%%%%%%%%%%%
%\section{Advantages / Disadvantages}

easy and much faster to get complex HDL code, such as image/video processing, through libraries like OpenCV.\\
However, it may be less optimized.\\
A developer still needs to have a strong hardware background, to be able to understand what is being generated in hardware from x lines of code in HLS/Cpp.
