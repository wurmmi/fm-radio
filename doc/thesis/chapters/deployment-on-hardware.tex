%%%%%%%%%%%%%%%%%%%%
\chapter{Deployment on Hardware}
\label{cha:DeploymentOnHardware}

This chapter describes the additional parts of implementation that are necessary, in order to run the implemented FM radio receiver on actual hardware.

%%%%%%%%%%%%%%%%%%%%
\section{GNU Radio}
\label{sec:deployment-on-hw:gnu-radio-devices}

The GNU Radio implementations, both receiver and transmitter, are used in combination with external hardware.
Two different hardware devices are used to interface with the real world, which are both supported by GNU Radio directly.
The specific devices are briefly described here.

%%%%%%%%%%%%%%%%%%%%
\subsection{Ettus Research USRP B200mini}

The USRP B200mini is a device that can be used to create a software-defined radio.
It resembles the size of a business card and is developed by Ettus Research.
A signal range from 70~MHz to 6~GHz can be covered by the front end.
The board further features a Xilinx Spartan-6 FPGA that is user-programmable, which makes the device very flexible for various applications with the received signal \cite{USRPb200Mini}.\\

GNU Radio supports this device with the blocks \textit{UHD: USRP Sink} and \textit{UHD: USRP Source}.
Therefore, it can be used as both a receiver and a transmitter.

%%%%%%%%%%%%%%%%%%%%
\subsection{RTL-SDR}

The RTL-SDR is a low-cost USB dongle, that was originally developed as a DVB-T TV tuner.
It is based on the RTL2832U chipset, which includes an 8-bit ADC and a digital signal processor.
The device delivers IQ data via the USB interface \cite{RTLSDR}.\\

\noindent
Based on its hardware, the RTL-SDR can only be used as a receiver.


%%%%%%%%%%%%%%%%%%%%
\section{FPGA Hardware Platform}

The ZedBoard is a development board that is developed by Xilinx.
It is specifically designed around the Xilinx Zynq-7000 SoC, which contains a dual-core ARM Cortex~A9 processor and a programmable logic fabric, the FPGA.
The board further offers a range of features like external DDR3 memory, various connectors, such as RJ45 ethernet, USB, 3.5~mm audio in- and output, HDMI, Pmod, as well as user interface hardware such as buttons, LEDs and a display.\\

The most important features for this project are the 3.5~mm audio line-out connector and the SD card port.
Additionally, the LEDs are used to display various status information during the runtime of the FM receiver.

\includepicture [0.75] [0] [H] {The ZedBoard PCB \cite{ZedBoard}.} {zedboard_photo} {img/zedboard_photo}

%%%%%%%%%%%%%%%%%%%%
\section{Product Design}

The implemented version of the FM radio receiver on the ZedBoard is a working prototype at this point.
The implementation does not support live data streaming.
Instead, antenna data is recorded to a file with the RTL-SDR hardware, using a regular PC.
The data is recorded as IQ data in the baseband.
The file is then stored onto an SD card, that is plugged into the ZedBoard.
The CPU on the SoC reads it and replays the antenna data to the FPGA.
In the future, this design may be extended to use an embedded Linux operating system.
In that way, the RTL-SDR could be connected to the ZedBoard PCB directly via USB, which would enable a live data stream from the antenna to the FPGA.
However, the chosen solution is sufficient for the aims of this thesis.

%%%%%%%%%%%%%%%%%%%%
\section{System-On-Chip Design}

Chapter~\ref{cha:Implementation} already described the development of the main IP core, which is the FM radio receiver in the VHDL and HLS variants.
In order to embed these IPs on the actual FPGA, or rather SoC hardware, a significant additional amount of implementation is neccessary.
This section describes these additional parts of the implementation, which specifically enable and support the deployment on the SoC.

%%%%%%%%%%%%%%%%%%%%
\subsection{Architecture}

The block diagram in Figure~\ref{fig:bd_impl_vivado} shows the implemented architecture that is used in the SoC.
The various design domains in, and around the SoC, are depicted in different visual representations.
The FPGA domain blocks are highlighted in gray, with the main IPs being surrounded by emphasized lines.
The processor system domain is shown as white blocks.
Blocks, which are external to the SoC, are drawn with dashed lines.

\includepicture [1.0] {Block diagram of the Vivado implementation.} {bd_impl_vivado} {img/draw.io/bd_impl_vivado}

%%%%%%%%%%%%%%%%%%%%
\subsection{Functionality}

The FM receiver IP is the main functionality of the design.
However, the design around it needs to function properly in order for the IP to work correctly, which is explained in the following passage.\\

The CPU is the main actor that determines any actions on the system.
First, it reads a file from the external SD card and stores this data in a dedicated area in the memory.
Next, the DMA is configured and its data transfer mode is initiated, so that the DMA starts to transfer data from the mentioned memory area towards the FPGA.
The DMA is configured in a mode that creates a continuous transfer, by looping through the data -- once the end is reached, the transfer starts at the beginning of the memory area.
Any data connection in the following processing chain is implemented as an AXI stream bus.
The first block after the DMA is an AXI Stream Switch.
It can switch its single input to be forwarded to either of the outputs.
The switch is configured by the CPU, to enable one output at a time.
Consequently, one of the FM radio receiver IPs is provided with data.
The IP processes this data to produce an audio output, which is delivered to another AXI Stream Switch.
It fulfills the function to either select the output of above or below IP, in coordination with the first switch, obviously.
The next block is an AXI Stream Broadcaster, which takes a single stream input and simply duplicates it at its output, so that the outputs are an exact copy of the input stream.
One of these duplicates is sourced to an AXI-Stream-To-I2S Converter, which generates the I2S signals for the external audio codec chip.
The second duplicate is connected to a FIFO, which simply stores the generated output data.
Once the FIFO is almost full, it asserts an interrupt to the CPU, which triggers it to read out the entire FIFO buffer.
The CPU then stores this data on the external SD card.
In that way, the loop of data is closed, which enables the verification of the respective IP's correct functionality.
Therefore, this system design can be described as a hardware-in-the-loop system, in the point of view of the FPGA IPs.
This ability refers to the knowledge gained with the SIMILAR concept, that is introduced in Section~\ref{sec:SimilarProcess}.
Specifically, this implemented system design enables the application of the steps \textit{Assess Performance} and \textit{Re-evaluate}, since the IPs' performance can be measured and their design can be re-evaluated.\\

It is to note, that Xilinx provides all the AXI infrastructure IPs, and thus the only self-implemented IPs are the FM radio receivers, as well as the I2S interface.
Nevertheless, the correct design and implementation of the architecture still needs to be formulated by the user.

%%%%%%%%%%%%%%%%%%%%
\subsection{Software}

The application software is running on a single core of the dual-core ARM Cortex~A9 CPU and is built upon a real-time operating system, namely FreeRTOS.
Xilinx provides a ported version of this RTOS with the installation of their software development kit, the Xilinx SDK.
The SDK also includes drivers to support various functionalities, such as support for accessing a FAT file system on the external SD card.\\

The respective software drivers for several IPs, which are used in the FPGA design, are automatically imported into the SDK project.
They are embedded in the so-called hardware definition file (HDF) that is generated by Vivado after the synthesis process, and is provided together with the bitstream file.
It is a file with the extension \textit{.hdf} and can be opened like a regular archive file, with any archive viewer.\\

Generally, the application software for this project is developed as an object-oriented \cplusplus\ program.
It uses common software development patterns and a class structure that fits the purpose.
The main tasks of the application are to configure several IPs in the FPGA, so that they can fulfill their respective purpose in the system, and to read the recorded data from the SD card, which is then streamed to the FPGA.
It also provides a basic user interface over the serial console, which a user can use to interact with the FM Radio Receiver product.
However, software design is not the main focus of this thesis, and thus, no further details are discussed here.
