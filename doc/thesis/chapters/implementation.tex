%%%%%%%%%%%%%%%%%%%%
\chapter{Implementation}
\label{cha:Implementation}

This chapter describes the most important details of each implementation variant.

\section{Digital Signal Processing Chain}

The theory behind the demodulation of an FM signal is already explained in Chapter \ref{cha:SignalProcessingTheory}.
The demodulation of an FM broadcast channel specifically, is described in Section \ref{sec:demodulation-of-broadcast-fm}.
In this section however, the DSP chain is described into more detail and with more information, as it is implemented in the various methods.\\

Fig.\ref{fig:bd_dsp_detailed} displays the detailed DSP chain as a block diagram. The following list explains implementation details about the respective DSP blocks.

\includepicture [1.0] {Detailed block diagram of the digital signal processing chain.} {bd_dsp_detailed} {img/bd_detailed_dsp}

\begin{itemize}
  \item \textbf{Data Source}\\
    Matlab is used to generate the data source file, which is used as an input to the system.
    An audio file is read and modulated as an FM channel signal.
  \item \textbf{FM Demodulator}\\
    The FM demodulator is implemented as a Baseband Delay Modulator, as described in \ref{sec:BasebandDelayDemodulator}.
  \item \textbf{De-Emphasis Filter}\\
    The above described Matlab source generation does not include an Emphasis filter.
    Thus, a de-emphasis filter is not implemented as it is not required.
  \item \textbf{Downsample}\\
    Downsampling is implemented in a simplified version, without an anti-aliasing filter.
    It is not required, because of the following explanation.
    At this point the sampling frequency is set to 960 kHz.
    The input is band-limited at 75 kHz, which is the maximum frequency in an FM channel, as explained in Section \ref{sec:FrequencySpectrumOfABroadcastFmChannel}.
    Thus, the input has an oversampling rate of 12.8, while downsampling is only performed with a factor of 8.
  \item \textbf{Filter Group Delay Compensation}\\
    FIR filters introduce a delay on the output signal, with respect to the input.
    This is due to the property of the FIR filters' linear phase response \cite{GaziOrhan2018UDSP}.
    Since phase is directly proportional to time, the term \textit{group delay} is often used.
    This delay needs to be accounted for and compensated in digital signal processing.

\end{itemize}

\section{Matlab Model}

\subsection{Emphasis and De-Emphasis}
show effects in spectrum, show how to design the filters

\subsection{Fixed Point Arithmetic}
not developed in actual fixed point, but close to hardware level algorithm



\subsection{Transmitter block diagram}

\section{GNU Radio}
  \subsection{Introduction}
  \subsection{Transmitter}
  \subsection{Receiver}

\section{VHDL (no shortcut)}
  \subsection{Testbench}
  \subsection{Channel Selection (IF to Channel-BB)}
  \subsection{Phase Detector}
  \subsection{other Elements}

  \section{High-Level Synthesis}
  \subsection{Testbench}
  \subsection{Channel Selection (IF to Channel-BB)}
  \subsection{Phase Detector}
  \subsection{other Elements}

\section{Common Testbench}
  \subsection{Architecture (same tb for VHDL and HLS-generated HDL)}
  \subsection{Framework cocotb, with ghdl compiler}
  Instantiate both HDL models in the testbench.\\
  Display a direct comparison of outputs in graphs. This is practical, since the cocotb framework runs in python and graphs can be generated using Python's matplotlib.

