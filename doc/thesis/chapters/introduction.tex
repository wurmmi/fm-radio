\chapter{Introduction}
\label{cha:Introduction}

%%%%%%%%%%%%%%%%%%%%%%%
\section{Motivation}

FM is a common RF signal that's available everywhere.\\
The demodulated signal is an audio signal, that can be listened to. This is subjectively more attractive than a generic data stream.

%%%%%%%%%%%%%%%%%%%%%%%
\section{Broadcast Radio}
Evolution from AM to FM because of certain advantages, etc.\\
Explain FM usage/existence nowadays (geographical; frequencies; devices; new standard DAB, etc)

\section{Goal}
put all knowledge that was accumulated during studies (HSD, ESD, mention subjects such as DSP) together in a project.\\
practical usage of DSP in an FPGA\\
develop an FM receiver and explain it, so that it can be followed in a tutorial with an affordable budget in hardware.\\
show multiple ways of how to implement the same thing, in different levels of abstraction. Weigh each ways' field of application, efficiency and applicability.


%%%%%%%%%%%%%%%%%%%%%%%
% NOTES:
%%%%%%%%%%%%%%%%%%%%%%%

Thesis Design Decisions: \\
\\
Main focus on system design. \\
\\
How to achieve the same result in 3 (4?) different levels of abstraction?

\begin{enumerate}
  \item GnuRadio (?)
  \item Matlab/Python
  \item HLS\\
      Implementation targeting Xilinx' ZedBoard.\\
      Optionally: \\
      Intel as a comparison (synth only, no HW) - "How platform-independent is HLS?"
  \item VHDL
\end{enumerate}
\vspace{.5 cm}

Compare:
- Implementation effort (difficulty)
- Implementation time
- Usefulness on Target Systems (uC, SoC, RasPi, etc)
- others?
