%%%%%%%%%%%%%%%%%%%%%%%%%%%%%%%%%%%%%%%%%%%%%%%%%%%%%%%%%%%%%%%%%%%%%%%%%%%%%%
\chapter{High-Level Synthesis}
\label{cha:HLS}

The main sources of information for this chapter are found in \cite{VivadoUgHLS}, TODO.

%%%%%%%%%%%%%%%%%%%%%%%%%%%%%%%%%%%%%%%%%%%%%%%%%%%%%%%%%%%%%%%%%%%%%%%%%%%%%%
\section{Introduction}

%%%%%%%%%%%%%%%%%%%%%%%%%%%%%%%%%%%%%%%%%%%%%%%%%%%%%%%%%%%%%%%%%%%%%%%%%%%%%%
\section{State Of The Art}

%%%%%%%%%%%%%%%%%%%%%%%%%%%%%%%%%%%%%%%%%%%%%%%%%%%%%%%%%%%%%%%%%%%%%%%%%%%%%%
\section{Functionality}
transform high-level code to HDL

%%%%%%%%%%%%%%%%%%%%%%%%%%%%%%%%%%%%%%%%%%%%%%%%%%%%%%%%%%%%%%%%%%%%%%%%%%%%%%
\section{Language Support}


%%%%%%%%%%%%%%%%%%%%%%%%%
\subsection{\cplusplus}

%%%%%%%%%%%%%%%%%%%%%%%%%
\subsection{SystemC}

In the latest Xilinx HLS tools 'Vitis HLS' and 'Vivado HLS', the language SystemC is already deprecated \cite{VivadoHlsSystemC}.\\

%%%%%%%%%%%%%%%%%%%%%%%%%%%%%%%%%%%%%%%%%%%%%%%%%%%%%%%%%%%%%%%%%%%%%%%%%%%%%%
\section{Coding}

%%%%%%%%%%%%%%%%%%%%%%%%%
\subsection{Compiler Directives (\#pragma's)}

%%%%%%%%%%%%%%%%%%%%%%%%%
\subsection{Data types}

%%%%%%%%%%%%%%%%%%%%%%%%%
\subsection{Functions}

%%%%%%%%%%%%%%%%%%%%%%%%%
\subsection{Loops}

%%%%%%%%%%%%%%%%%%%%%%%%%
\subsection{Conditional statements}

%%%%%%%%%%%%%%%%%%%%%%%%%%%%%%%%%%%%%%%%%%%%%%%%%%%%%%%%%%%%%%%%%%%%%%%%%%%%%%
\section{Testbench}

An HLS testbench can be executed in different levels of abstraction, which is a major difference between a VHDL testbench and an HLS testbench.
The most important impact thereof is the execution speed - the time it takes to simulate a number of test cases on the DUT.

%%%%%%%%%%%%%%%%%%%%%%%%%
\subsection{C Simulation}

In C Simulation, the HLS compiler is used to create an executable binary that includes the testbench code, as well as the IP design of the DUT.
This binary can be run just like any other program.
The most important note here is, that the HLS code is \textit{not} transformed into RTL at this point.
Instead, the HLS code is treated as a regular \cplusplus\ code, that uses some libraries, like the fixed-point library.
This means, that the HLS IP cores' top-level function is used as a regular function call in the testbench.\\

The main advantage of this type of simulation is the major increase of execution speed.
The reason for this speed-up is the abstraction of timing, such as clock cycles or similar constraints.
In fact, there is absolutely no timing information that needs to be taken into account in this simulation.

%%%%%%%%%%%%%%%%%%%%%%%%%
\subsection{C/RTL Co-Simulation}

This type of simulation actually transforms the HLS IP code into RTL code, such as Verilog or VHDL, for the testbench.
This is done in three steps.\\

\begin{itemize}
  \item \textbf{Create input vectors}\\
  In this first step, the above explained, regular C simulation is executed.
  Hereby, the inputs that are sent to the DUT are stored as 'input vectors'.

  \item \textbf{Perform RTL simulation}\\
  Next, the previously generated input vectors are applied to the DUT.
  At this point, the DUT is already translated into RTL code, so the code that is used here is either VHDL or Verilog code.
  The default simulator is the Vivado Simulator (XSim).
  However, various other simulators like ModelSim are supported.
  The generated outputs of this step are stored as 'output vectors'.
  \item \textbf{Verify the results}\\
  The final verification step sources the output vectors back to the C testbench, which compares the outputs with the expected results.\\
\end{itemize}

\noindent
This type of simulation takes a significantly longer time to execute, because timing information now actually needs to be considered for the RTL code.

%%%%%%%%%%%%%%%%%%%%%%%%%%%%%%%%%%%%%%%%%%%%%%%%%%%%%%%%%%%%%%%%%%%%%%%%%%%%%%
\section{Advantages / Disadvantages}

easy and much faster to get complex HDL code, such as image/video processing, through libraries like OpenCV. However, it is less optimized, and a developer still needs to have a strong hardware background, to be able to understand what is being generated in hardware from x lines of code in HLS/Cpp.
