%%%%%%%%%%%%%%%%%%%%
\chapter{System Architecture and Concept}
\label{cha:SystemArchitectureAndConcept}

This chapter introduces the FM radio receiver project that is implemented as a major part of this thesis.
It specifically talks about the system level approach that is taken in the implementation of the project.
The main concept, as well as the architecture are explained.

%%%%%%%%%%%%%%%%%%%%
\section{Overview}

In the development of a project that includes multiple functional parts, it is always of advantage to start with a system level overview.
Major decisions can be made at this stage, before any implementation has begun.
This can have a direct impact on the efficiency of the entire development, which concequently has an impact on the final quality of the product.
A thorough system level design can also prevent from issues that would otherwise arise while the development is already ongoing.
As an example it could turn out that one of the functional parts is not able to fit into the integrated system, in the way it is implemented.
This leads to the necessity to re-engineer this entire functional block.
As a result, it takes more effort to implement the functionality, which directly correlates with the development cost and the time-to-market, in order to deliver the final product.
Summing up, a comprehensive system level architecture and concept can pave the way for a successful implementation of a project from start to finish.\\

In order to create a system level design, considerations around the high-level, final goal of the product are a good starting point.
This includes the definition of its functional range.
Once this is done, the target system or target hardware should be considered, because there are various ways to implement a certain functionality.
All of these variants will eventually result in the same output, but will, however, heavily differ in their implementation effort, efficiency, or their applicability for the products' aim in general.
Thus, things like the available hardware platform, implementation time, degree of optimization, output quality, and similar things are important factors at this stage.
Of course, there are many more factors to consider when a product is to be developed.
However, only a subset of these is covered here, since this chapter is focused on how this thesis' accompanying project was developed and is thus written in that context.\\

%%%%%%%%%%%%%%%%%%%%
\section{The Project}

The aim of this thesis is develop a system architecture and concept, that allows the implementation of an FM radio receiver in multiple different ways, while providing an elegant way to compare the different solutions.
The final goal thereby is always be the same, that is: Listen to the music of a radio station that is transmitted over FM broadcast radio.\\

\noindent
A very high-level overview of the entire system is given by the block diagram in Fig.\ref{fig:bd_system_overview_highlevel}.
It shows the front end, which processes the received antenna signal, so that it can be used in the FM radio receiver block.
The radio receiver itself is implemented in multiple different variants.
The produced audio output is sent to a speaker, since the ultimate goal is to listen to a radio stations' audio signal.

\includepicture [1.0] [0] [H] {High-level overview of the system.} {bd_system_overview_highlevel} {img/draw.io/bd_system_overview_highlevel}

In the context of a system level design approach, several steps are necessary to find a solution that is ready to be implemented.
As a very first step, the inner workings of an FM receiver need to be understood.
This mainly regards the signal processing theory, to be able to decode the signal that is received by the antenna.
Chapter \ref{cha:SignalProcessingTheory} describes this part into detail.

As already mentioned above, multiple ways of technology and implementation can lead to a similar result.
Thus, the next step is to define a set of possible ways which will actually be implemented.
For this thesis, the decision was made to implement the FM radio receiver by using the following methods.

\begin{itemize}
  \item \textbf{GNU Radio}\\
      GNU Radio is a free and open-source software development toolkit to implement a software-defined radio.
      The implementation is done by connecting functional blocks in a block design.
      This abstracts the inner workings of each functional block and thus provides an implementation approach on a very high level of abstraction.
      GNU Radio supports interfaces to external RF hardware \cite{SoftwareGnuRadio}.
  \item \textbf{Matlab}\\
      Matlab is a software platform that can work with matrices and compute numerical solutions.
      Matlab also includes various toolboxes, that support developers with complex calculations and tasks on datasets.
      This includes a range of toolboxes for signal processing, such as filter designers or FFT functions, as well as convenient ways to visualize data \cite{SoftwareMatlab}.
  \item \textbf{FPGA: \cplusplus\ High-Level Synthesis}\\
      In order to use an FPGA, a hardware design needs to be developed for it.
      This can be done using high-level synthesis.
      This thesis focuses in \cplusplus and does not implement a SystemC variant, because Xilinx has deprecated SystemC support for synthesis in their latest HLS tools.
      This, and the details of High-level synthesis (HLS) are explained in Chapter \ref{cha:HLS}.

  \item \textbf{FPGA: VHDL}\\
      Similar to the HLS approach, this is a variant to write a hardware design for an FPGA.
      The language VHDL is used in this approach.
\end{itemize}

In order to implement an FM radio receiver, all of the listed methods require deep knowledge about the neccessary details of signal processing.
However, the implementation of the various different methods require more or less knowledge and effort on the exact underlying implementation of each signal processing part.
This depends on their respective level of abstraction.

%-  implementation of a model and DSP concept in Matlab \\
%-  implementation in VHDL and HLS \\
%-  both use the same testbench source \\
%-  both are integrated into FPGA
%-  the FPGA output is read back into the firmware, for analysis \\

%%%%%%%%%%%%%%%%%%%%
\section{System Concept}

The overall system concept has to be developed in a way, to fit the respective projects' needs and requirements.
There should be a clear structure, in order to help to create a working system successfully.
This structure should include a description of the data source, the data processing blocks and stages, commonly used parameters, as well as a strategy to verify the correct functionality of the system.
A good system structure tries to share as many parts as possible in the system.
This is to minimize the points of failure, as well as to reduce the effort of implementation.\\

In the apparent projects' case, a single block of the system is implemented in multiple ways, which should, however, all compute an equal output.
Thus, it is of advantage to define the data source to be common for all of these variants, to create a unified situation at the input.
Additionally, the data source may be chosen to be recorded, to have a known, reproducable set of input data.
This significantly eases the verification of the functionality in the final step, since all variants are expected to create the same output in the optimum, fully functioning case.
Not only the data source is important to be common, but also the way of verification.
In order to achieve that, the verification block needs to be able to compare multiple data outputs with the verification data, which is chosen to serve as the reference.\\

An example of a system concept and its architecture is developed and shown in Fig.\ref{fig:bd_system_overview_detailed}.
Here, a common data source is provided by the block called \textit{Front End}.
It records data from an antenna and saves it to a file, in a format that the consequent blocks require at their input.
Therefore, a reproducable input is ensured.
Within the \textit{FM Radio Receiver} block, all the different implementation variants are shown in parallel.
They all work on this common input source and generate their respective output.
This output data is again written to a file.
Hereby it is important to note, that the output of the \textit{Matlab} implementation is chosen to be the 'golden reference model' of correct output.
Its output is used as the verification data later in the process.
This is because Matlab was used to research decoding algorithms and to develop the DSP chain, which is used in the GNU Radio, HLS and VHDL implementations.
In addition to the verification data, the Matlab model also writes common parameters to a file.
These parameters are directly used in the other implementations, to ensure a common setup.
The final block is the \textit{Verification} entity, which compares the output files of all the variants, with respect to the chosen model parameters.\\

A major point in this system concept is, that it works independently of the respective implementation methodology.
This is especially important, since there is no possible way to run all these different variants in a single verification environment.
Instead, each methodology writes an output file, which is a supported functionality that is available anywhere.

\includepicture [1.0] [0] {Block diagram of the overall system concept.} {bd_system_overview_detailed} {img/draw.io/bd_system_overview_detailed}


%%%%%%%%%%%%%%%%%%%%%
%\section{Test Environment}
