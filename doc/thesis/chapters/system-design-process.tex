%%%%%%%%%%%%%%%%%%%%
\chapter{System Design Process}
\label{cha:SystemDesignProcess}

The process of system design is a central topic in the development of any product.
This chapter describes general techniques for system design.
However, they can also be directly applied to FPGA design.
Additionally, tools that are specifically used in FPGA design are introduced.

%%%%%%%%%%%%%%%%%%%%%%%%%%%%%%%%%%%%%%%%%%%%%%%%%%%%%%%%%%%%%%%%%%%%%%%%%%%%%%
\section{General}

The following sections describe a generalized approach to system design.
A process called SIMILAR is explained, as well as selected topics that are relevant.
The researched information for this section is mainly based on \cite{BahillA.Terry2017TDiS}.

%%%%%%%%%%%%%%%%%%%%%%%%%
\subsection{The SIMILAR Process}

The SIMILAR process provides a general approach to problem solving.
It can be applied to any kind of problem that can be of technical, as well as non-technical nature, in order to successfully achieve the specified requirements.
The concept was developed by A.T. Bahill and B. Gissing, who evaluated and compared multiple existing processes to find their similarities.\\

The acronym SIMILAR consists of the following components.

\begin{itemize}
  \item \textbf{S}tate the Problem
  \item \textbf{I}nvestigate Alternatives
  \item \textbf{M}odel the System
  \item \textbf{I}ntegrate Components
  \item \textbf{L}aunch the System
  \item \textbf{A}ssess Performance
  \item \textbf{R}e-evaluate
\end{itemize}

Each of these terms represents a separate step in the process diagram shown in Fig.\ref{fig:similar_process}.
The steps can be seen as iterative and parallel, as the process runs during the product development cycle.

\includepicture [0.8] [0] {The SIMILAR process in a graphical representation.} {similar_process} {img/similar_process}

%%%%%
\subsubsection{State the Problem}

The very first step of the process is to state the problem that is to be solved.
This can be the initial definition of a new product, but can also be the definition of an improvement that is to be made for an existing product.
Because of that, input for the problem statement can be generated by various parties, such as customers, product developers, regulatory agencies or manufacturers, just to name a few.
In the process diagram, all these possible input parties are labeled as stakeholders.\\

An important guideline for the specification of this problem statement is, that each requirement is descibed in the way of \textit{what} needs to be done, not \textit{how} it needs to be done.\\

The problem statement is updated and improved by the re-evaluate step, which is described later.

%%%%%
\subsubsection{Investigate Alternatives}



%%%%%
\subsubsection{Model the System}

%%%%%
\subsubsection{Integrate Components}

%%%%%
\subsubsection{Launch the System}

%%%%%
\subsubsection{Assess Performance}

%%%%%
\subsubsection{Re-evaluate}


%%%%%%%%%%%%%%%%%%%%%%%%%
\subsection{Design for Reuse}
%see pg. 51

%%%%%%%%%%%%%%%%%%%%%%%%%
\subsection{The Problem Statement}

%%%%%%%%%%%%%%%%%%%%%%%%%
\subsection{Concept Exploration}

%%%%%%%%%%%%%%%%%%%%%%%%%
\subsection{Validation and Verification}

%%%%%%%%%%%%%%%%%%%%%%%%%%%%%%%%%%%%%%%%%%%%%%%%%%%%%%%%%%%%%%%%%%%%%%%%%%%%%%
\section{System Design Targeting FPGAs}
The research for this chapter is mainly based on \cite{GesslerRalf2014EES}.
%https://link-springer-com.fhooe.idm.oclc.org/content/pdf/10.1007%2F978-3-319-26408-0.pdf
%chapter 3

%%%%%%%%%%%%%%%%%%%%%%%%%
\subsection{V-Modell}

%applies to general software development, but can directly be mapped to FPGA development.

%Abbildung 3.5 in
%https://link-springer-com.fhooe.idm.oclc.org/content/pdf/10.1007%2F978-3-8348-2080-8.pdf

%%%%%%%%%%%%%%%%%%%%%%%%%
\subsection{High-Level Synthesis}
%\cplusplus\\
%SystemC (deprecated for synthesis by Xilinx)\\

A comprehensive list of HLS tools can be found in Section 3.3 of
%https://link-springer-com.fhooe.idm.oclc.org/content/pdf/10.1007%2F978-3-319-26408-0.pdf

%%%%%%%%%%%%%%%%%%%%%%%%%
\subsection{Implementation with Tool Support}
%https://link-springer-com.fhooe.idm.oclc.org/content/pdf/10.1007%2F978-3-8348-2080-8.pdf

%Matlab Simulink System Generator\\
%Matlab Simulink HDL Coder\\

%%%%%%%%%%%%%%%%%%%%%%%%%
\subsection{Direct Implementation}

%direct, or 'manual' implementation\\
%VHDL/Verilog\\
