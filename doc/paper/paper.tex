
%% bare_conf.tex
%% V1.4b
%% 2015/08/26
%% by Michael Shell
%% See:
%% http://www.michaelshell.org/
%% for current contact information.
%%
%% This is a skeleton file demonstrating the use of IEEEtran.cls
%% (requires IEEEtran.cls version 1.8b or later) with an IEEE
%% conference paper.
%%
%% Support sites:
%% http://www.michaelshell.org/tex/ieeetran/
%% http://www.ctan.org/pkg/ieeetran
%% and
%% http://www.ieee.org/

%%*************************************************************************
%% Legal Notice:
%% This code is offered as-is without any warranty either expressed or
%% implied; without even the implied warranty of MERCHANTABILITY or
%% FITNESS FOR A PARTICULAR PURPOSE!
%% User assumes all risk.
%% In no event shall the IEEE or any contributor to this code be liable for
%% any damages or losses, including, but not limited to, incidental,
%% consequential, or any other damages, resulting from the use or misuse
%% of any information contained here.
%%
%% All comments are the opinions of their respective authors and are not
%% necessarily endorsed by the IEEE.
%%
%% This work is distributed under the LaTeX Project Public License (LPPL)
%% ( http://www.latex-project.org/ ) version 1.3, and may be freely used,
%% distributed and modified. A copy of the LPPL, version 1.3, is included
%% in the base LaTeX documentation of all distributions of LaTeX released
%% 2003/12/01 or later.
%% Retain all contribution notices and credits.
%% ** Modified files should be clearly indicated as such, including  **
%% ** renaming them and changing author support contact information. **
%%*************************************************************************


% *** Authors should verify (and, if needed, correct) their LaTeX system  ***
% *** with the testflow diagnostic prior to trusting their LaTeX platform ***
% *** with production work. The IEEE's font choices and paper sizes can   ***
% *** trigger bugs that do not appear when using other class files.       ***                          ***
% The testflow support page is at:
% http://www.michaelshell.org/tex/testflow/



\documentclass[conference]{IEEEtran}
% Some Computer Society conferences also require the compsoc mode option,
% but others use the standard conference format.
%
% If IEEEtran.cls has not been installed into the LaTeX system files,
% manually specify the path to it like:
% \documentclass[conference]{../sty/IEEEtran}





% Some very useful LaTeX packages include:
% (uncomment the ones you want to load)


% *** MISC UTILITY PACKAGES ***
%
%\usepackage{ifpdf}
% Heiko Oberdiek's ifpdf.sty is very useful if you need conditional
% compilation based on whether the output is pdf or dvi.
% usage:
% \ifpdf
%   % pdf code
% \else
%   % dvi code
% \fi
% The latest version of ifpdf.sty can be obtained from:
% http://www.ctan.org/pkg/ifpdf
% Also, note that IEEEtran.cls V1.7 and later provides a builtin
% \ifCLASSINFOpdf conditional that works the same way.
% When switching from latex to pdflatex and vice-versa, the compiler may
% have to be run twice to clear warning/error messages.






% *** CITATION PACKAGES ***
%
\usepackage{cite}
% cite.sty was written by Donald Arseneau
% V1.6 and later of IEEEtran pre-defines the format of the cite.sty package
% \cite{} output to follow that of the IEEE. Loading the cite package will
% result in citation numbers being automatically sorted and properly
% "compressed/ranged". e.g., [1], [9], [2], [7], [5], [6] without using
% cite.sty will become [1], [2], [5]--[7], [9] using cite.sty. cite.sty's
% \cite will automatically add leading space, if needed. Use cite.sty's
% noadjust option (cite.sty V3.8 and later) if you want to turn this off
% such as if a citation ever needs to be enclosed in parenthesis.
% cite.sty is already installed on most LaTeX systems. Be sure and use
% version 5.0 (2009-03-20) and later if using hyperref.sty.
% The latest version can be obtained at:
% http://www.ctan.org/pkg/cite
% The documentation is contained in the cite.sty file itself.






% *** GRAPHICS RELATED PACKAGES ***
%
\ifCLASSINFOpdf
  \usepackage[pdftex]{graphicx}
  % declare the path(s) where your graphic files are
  % \graphicspath{{../pdf/}{../jpeg/}}
  % and their extensions so you won't have to specify these with
  % every instance of \includegraphics
  % \DeclareGraphicsExtensions{.pdf,.jpeg,.png}
\else
  % or other class option (dvipsone, dvipdf, if not using dvips). graphicx
  % will default to the driver specified in the system graphics.cfg if no
  % driver is specified.
  % \usepackage[dvips]{graphicx}
  % declare the path(s) where your graphic files are
  % \graphicspath{{../eps/}}
  % and their extensions so you won't have to specify these with
  % every instance of \includegraphics
  % \DeclareGraphicsExtensions{.eps}
\fi
% graphicx was written by David Carlisle and Sebastian Rahtz. It is
% required if you want graphics, photos, etc. graphicx.sty is already
% installed on most LaTeX systems. The latest version and documentation
% can be obtained at:
% http://www.ctan.org/pkg/graphicx
% Another good source of documentation is "Using Imported Graphics in
% LaTeX2e" by Keith Reckdahl which can be found at:
% http://www.ctan.org/pkg/epslatex
%
% latex, and pdflatex in dvi mode, support graphics in encapsulated
% postscript (.eps) format. pdflatex in pdf mode supports graphics
% in .pdf, .jpeg, .png and .mps (metapost) formats. Users should ensure
% that all non-photo figures use a vector format (.eps, .pdf, .mps) and
% not a bitmapped formats (.jpeg, .png). The IEEE frowns on bitmapped formats
% which can result in "jaggedy"/blurry rendering of lines and letters as
% well as large increases in file sizes.
%
% You can find documentation about the pdfTeX application at:
% http://www.tug.org/applications/pdftex





% *** MATH PACKAGES ***
%
\usepackage{amsmath}
% A popular package from the American Mathematical Society that provides
% many useful and powerful commands for dealing with mathematics.
%
% Note that the amsmath package sets \interdisplaylinepenalty to 10000
% thus preventing page breaks from occurring within multiline equations. Use:
%\interdisplaylinepenalty=2500
% after loading amsmath to restore such page breaks as IEEEtran.cls normally
% does. amsmath.sty is already installed on most LaTeX systems. The latest
% version and documentation can be obtained at:
% http://www.ctan.org/pkg/amsmath





% *** SPECIALIZED LIST PACKAGES ***
%
%\usepackage{algorithmic}
% algorithmic.sty was written by Peter Williams and Rogerio Brito.
% This package provides an algorithmic environment fo describing algorithms.
% You can use the algorithmic environment in-text or within a figure
% environment to provide for a floating algorithm. Do NOT use the algorithm
% floating environment provided by algorithm.sty (by the same authors) or
% algorithm2e.sty (by Christophe Fiorio) as the IEEE does not use dedicated
% algorithm float types and packages that provide these will not provide
% correct IEEE style captions. The latest version and documentation of
% algorithmic.sty can be obtained at:
% http://www.ctan.org/pkg/algorithms
% Also of interest may be the (relatively newer and more customizable)
% algorithmicx.sty package by Szasz Janos:
% http://www.ctan.org/pkg/algorithmicx




% *** ALIGNMENT PACKAGES ***
%
%\usepackage{array}
% Frank Mittelbach's and David Carlisle's array.sty patches and improves
% the standard LaTeX2e array and tabular environments to provide better
% appearance and additional user controls. As the default LaTeX2e table
% generation code is lacking to the point of almost being broken with
% respect to the quality of the end results, all users are strongly
% advised to use an enhanced (at the very least that provided by array.sty)
% set of table tools. array.sty is already installed on most systems. The
% latest version and documentation can be obtained at:
% http://www.ctan.org/pkg/array


% IEEEtran contains the IEEEeqnarray family of commands that can be used to
% generate multiline equations as well as matrices, tables, etc., of high
% quality.




% *** SUBFIGURE PACKAGES ***
%\ifCLASSOPTIONcompsoc
%  \usepackage[caption=false,font=normalsize,labelfont=sf,textfont=sf]{subfig}
%\else
%  \usepackage[caption=false,font=footnotesize]{subfig}
%\fi
% subfig.sty, written by Steven Douglas Cochran, is the modern replacement
% for subfigure.sty, the latter of which is no longer maintained and is
% incompatible with some LaTeX packages including fixltx2e. However,
% subfig.sty requires and automatically loads Axel Sommerfeldt's caption.sty
% which will override IEEEtran.cls' handling of captions and this will result
% in non-IEEE style figure/table captions. To prevent this problem, be sure
% and invoke subfig.sty's "caption=false" package option (available since
% subfig.sty version 1.3, 2005/06/28) as this is will preserve IEEEtran.cls
% handling of captions.
% Note that the Computer Society format requires a larger sans serif font
% than the serif footnote size font used in traditional IEEE formatting
% and thus the need to invoke different subfig.sty package options depending
% on whether compsoc mode has been enabled.
%
% The latest version and documentation of subfig.sty can be obtained at:
% http://www.ctan.org/pkg/subfig




% *** FLOAT PACKAGES ***
%
%\usepackage{fixltx2e}
% fixltx2e, the successor to the earlier fix2col.sty, was written by
% Frank Mittelbach and David Carlisle. This package corrects a few problems
% in the LaTeX2e kernel, the most notable of which is that in current
% LaTeX2e releases, the ordering of single and double column floats is not
% guaranteed to be preserved. Thus, an unpatched LaTeX2e can allow a
% single column figure to be placed prior to an earlier double column
% figure.
% Be aware that LaTeX2e kernels dated 2015 and later have fixltx2e.sty's
% corrections already built into the system in which case a warning will
% be issued if an attempt is made to load fixltx2e.sty as it is no longer
% needed.
% The latest version and documentation can be found at:
% http://www.ctan.org/pkg/fixltx2e


%\usepackage{stfloats}
% stfloats.sty was written by Sigitas Tolusis. This package gives LaTeX2e
% the ability to do double column floats at the bottom of the page as well
% as the top. (e.g., "\begin{figure*}[!b]" is not normally possible in
% LaTeX2e). It also provides a command:
%\fnbelowfloat
% to enable the placement of footnotes below bottom floats (the standard
% LaTeX2e kernel puts them above bottom floats). This is an invasive package
% which rewrites many portions of the LaTeX2e float routines. It may not work
% with other packages that modify the LaTeX2e float routines. The latest
% version and documentation can be obtained at:
% http://www.ctan.org/pkg/stfloats
% Do not use the stfloats baselinefloat ability as the IEEE does not allow
% \baselineskip to stretch. Authors submitting work to the IEEE should note
% that the IEEE rarely uses double column equations and that authors should try
% to avoid such use. Do not be tempted to use the cuted.sty or midfloat.sty
% packages (also by Sigitas Tolusis) as the IEEE does not format its papers in
% such ways.
% Do not attempt to use stfloats with fixltx2e as they are incompatible.
% Instead, use Morten Hogholm'a dblfloatfix which combines the features
% of both fixltx2e and stfloats:
%
% \usepackage{dblfloatfix}
% The latest version can be found at:
% http://www.ctan.org/pkg/dblfloatfix




% *** PDF, URL AND HYPERLINK PACKAGES ***
%
\usepackage{url}
% url.sty was written by Donald Arseneau. It provides better support for
% handling and breaking URLs. url.sty is already installed on most LaTeX
% systems. The latest version and documentation can be obtained at:
% http://www.ctan.org/pkg/url
% Basically, \url{my_url_here}.




% *** Do not adjust lengths that control margins, column widths, etc. ***
% *** Do not use packages that alter fonts (such as pslatex).         ***
% There should be no need to do such things with IEEEtran.cls V1.6 and later.
% (Unless specifically asked to do so by the journal or conference you plan
% to submit to, of course. )


% correct bad hyphenation here
\hyphenation{op-tical net-works semi-conduc-tor}


\begin{document}
%
% paper title
% Titles are generally capitalized except for words such as a, an, and, as,
% at, but, by, for, in, nor, of, on, or, the, to and up, which are usually
% not capitalized unless they are the first or last word of the title.
% Linebreaks \\ can be used within to get better formatting as desired.
% Do not put math or special symbols in the title.
\title{Signal Processing to Demodulate FM Radio}


% author names and affiliations
% use a multiple column layout for up to three different
% affiliations
\author{\IEEEauthorblockN{Michael Wurm}
\IEEEauthorblockA{University of Applied Sciences Upper Austria\\
Campus Hagenberg\\
Softwarepark 11, 4232 Hagenberg\\
Email: michael.wurm@students.fh-hagenberg.at}
%\and
%\IEEEauthorblockN{Homer Simpson}
%\IEEEauthorblockA{Twentieth Century Fox\\
%Springfield, USA\\
%Email: homer@thesimpsons.com}
%\and
%\IEEEauthorblockN{James Kirk\\ and Montgomery Scott}
%\IEEEauthorblockA{Starfleet Academy\\
%San Francisco, California 96678--2391\\
%Telephone: (800) 555--1212\\
%Fax: (888) 555--1212}
}

% conference papers do not typically use \thanks and this command
% is locked out in conference mode. If really needed, such as for
% the acknowledgment of grants, issue a \IEEEoverridecommandlockouts
% after \documentclass

% for over three affiliations, or if they all won't fit within the width
% of the page, use this alternative format:
%
%\author{\IEEEauthorblockN{Michael Shell\IEEEauthorrefmark{1},
%Homer Simpson\IEEEauthorrefmark{2},
%James Kirk\IEEEauthorrefmark{3},
%Montgomery Scott\IEEEauthorrefmark{3} and
%Eldon Tyrell\IEEEauthorrefmark{4}}
%\IEEEauthorblockA{\IEEEauthorrefmark{1}School of Electrical and Computer Engineering\\
%Georgia Institute of Technology,
%Atlanta, Georgia 30332--0250\\ Email: see http://www.michaelshell.org/contact.html}
%\IEEEauthorblockA{\IEEEauthorrefmark{2}Twentieth Century Fox, Springfield, USA\\
%Email: homer@thesimpsons.com}
%\IEEEauthorblockA{\IEEEauthorrefmark{3}Starfleet Academy, San Francisco, California 96678-2391\\
%Telephone: (800) 555--1212, Fax: (888) 555--1212}
%\IEEEauthorblockA{\IEEEauthorrefmark{4}Tyrell Inc., 123 Replicant Street, Los Angeles, California 90210--4321}}




% use for special paper notices
%\IEEEspecialpapernotice{(Invited Paper)}




% make the title area
\maketitle

% As a general rule, do not put math, special symbols or citations
% in the abstract
\begin{abstract}
  This paper describes the necessary parts of digital signal processing, in order to be able to listen to a radio station that is broadcast in the FM frequency band.
  Fundamental knowledge about digital signal processing is presented.
  The processes of signal modulation up to radio frequency and down to baseband are described, as well as frequency modulation as a modulation type to transmit data.
  A single FM radio channel in the FM radio band consists of multiple parts in the frequency spectrum, which is an important factor to be able to demodulate an audio stream correctly.
  This demodulation itself can be done in different algorithms, of which a selection is presented.
  The entire demodulation chain is implemented in a Matlab simulation script, to demonstrate the functionality of an FM radio receiver.

  %The aim of this master thesis is to compare an FM radio receiver implementation in HLS versus manual VHDL. Multiple metrics, such as time effort in implementation, hardware utilization in the FPGA, or efficiency are taken into account. As a first step, the receiver structure is simulated using Matlab to evaluate different architectures and techniques to demodulate a FM radio signal. This evaluation is especially targeting a setup that requires the least possible amount of analog hardware. Once a suitable strategy is found, a prototype is implemented in hardware on a FPGA.\\
\end{abstract}

\begin{IEEEkeywords}
%  Audio signal processing, FM radio frequency, FPGA, HLS, VHDL, Matlab
  Audio signal processing, FM radio frequency, Demodulation, Matlab, Simulation
\end{IEEEkeywords}


% For peer review papers, you can put extra information on the cover
% page as needed:
% \ifCLASSOPTIONpeerreview
% \begin{center} \bfseries EDICS Category: 3-BBND \end{center}
% \fi
%
% For peerreview papers, this IEEEtran command inserts a page break and
% creates the second title. It will be ignored for other modes.
\IEEEpeerreviewmaketitle

%%%%%%%%%%%%%%%%%%%%%%%%%%%%%%%%%%%%%%%%%%%%%%%%%%%%%%%%%%%%%%%%%%%%%%%
\section{Introduction}

  Frequency modulation (FM) is a widely used standard to transmit data streams.
  The probably best known usecase therefore is commercial broadcast radio, where an audio stream is transmitted.
  Devices to receive these streams are available starting from affordable low prices to the public.
  Since this is such a common technology which so many people are using on a daily basis, and which is included in virtually every automobile, a more detailed look into this signal transmission system is very interesting.

  Multiple courses in the program Embedded Systems Design at the University of Applied Sciences Upper Austria, Campus Hagenberg, cover the topic of digital signal processing.
  This paper tries to accumulate all this knowledge in a single paper and give an overview about the FM radio technology.
  A thesis for the Masters' program will be elaborated based on this topic.

%%%%%%%%%%%%%%%%%%%%%%%%%%%%%%%%%%%%%%%%%%%%%%%%%%%%%%%%%%%%%%%%%%%%%%%
%\section{Signal Processing Theory}
%
%  A comprehensive set of signal processing techniques is required, in order to %achieve a data transmission with frequency modulation.
%  In traditional commercial senders and receivers...
%
%  \subsection{Modulation from Baseband to RF}
%
%
%  \subsection{Modulation from RF to Baseband}
%  \subsection{Frequency Modulation}

  \section{Frequency Modulation}
    Frequency modulation (FM) is a modulation method, in which the information content is encoded in the frequency of a signal.
    To be exact, the information is encoded in the frequencies' deviation from the carrier frequency.
    The technology is widely used in FM broadcast radio.
    One of its main advantages over amplitude modulation (AM) is the higher signal-to-noise ration (SNR).
    This is achieved, because of the constant amplitude in FM, which makes it less sensitive to additive noise in the transmission channel.
    Additionally, this fact makes it easier for a receiver to synchronize on the received signal \cite{FM_Maths_Info_1}.

  \subsection{Mathematical Description}
    The signal of an FM baseband signal can be described in the following formulas.
    As already mentioned in earlier in this paper, FM encodes the information in the frequency.
    Since the frequency has a direct relationship to the angle, FM belongs to the group of angle modulated signals.

    The instantanious frequency $f_i$ can be described as
    \begin{equation}
      f_i = f_c + \Delta f \cdot m(t)
    \end{equation}
    where $f_c$ is the carrier frequency, $\Delta f$ is the maximum frequency deviation and $m(t)$ is the information or message signal that is to be transmitted.
    Simply looking at this equation, the frequency deviation, which is also called the swing, varies in the range between $f_c \pm \Delta f$.

    Considering the relationship between angular frequency and angle and after some simple substitutions, which will not be described into detail here, the equation for a generic frequency modulated wave is the following.
    \begin{equation}
      s(t) = A_c\ cos \Big( 2 \pi f_c t + 2 \pi k_f \int m(t) dt \Big)
      \label{equ_fm_func}
    \end{equation}
    Where $A_c$ is the amplitude of the resulting FM signal and $k_f$ is the frequency sensitivity factor.

    An example application of FM is transmitting audio streams.
    An audio stream can be described as a cosine wave.
    If the information signal that is to be transmitted is a cosine wave in the form of
    \begin{equation}
      m(t) = A_m\ cos(2 \pi f_m t)
    \end{equation}
    the final equation for an FM signal transforms into the following formula.
    \begin{equation}
      y(t) = A_c\ cos \Big(2 \pi f_c t + \beta\ sin(2 \pi f_m t)\Big )
    \end{equation}

    The variable $\beta$ is called the modulation index,
    \begin{equation}
      \beta = \frac{\Delta f}{f_m} = \frac{k_f A_m}{f_m}
    \end{equation}
    where $f_m$ is the highest existing frequency, or the bandwidth, of the information signal.
    Several formulas, as well as their derivations were described according to \cite{FM_Maths_Info_1}\cite{FMMaths2}.

  \subsection{Frequency Band}
    The frequency band that is used for FM broadcasting is defined worldwide and spans from 87.5 to 108 MHz.
    However, some countries only partially use the band \cite{itu_regulations}.
    The range is located within the so-called Very-High-Frequency (VHF) band.
    In terms of usage, the band is open to anyone.
    This means, that a transmitter may use the specified frequency range freely.
    Austria allowed the legal usage in 2006.
    However, the transmission power needs to be limited, so that neighboring receivers are not disturbed in receiving any existing channels.
    As an example, the German Federal Network Agency specifies the power limit as 50 nW, or -43 dBm, of effective radiated power (ERP).
    In a practical usecase, this means that a transmitter needs to select a channel, or frequency range that is not already occupied by an official sender.
    If a free range is found, the transmitter may start sending an FM signal with the mentioned power limitations \cite{ebu_fm_modulators_in_europe_power_regulations}.

    %https://www.ris.bka.gv.at/GeltendeFassung.wxe?Abfrage=Bundesnormen&Gesetzesnummer=20008807
    %https://www.rtr.at/medien/was_wir_tun/frequenzverwaltung/Frequenzen.en.html

  \subsection{Channel Frequency Spectrum}
    The entire FM band frequency range divides into multiple channels which can be used.
    In Europe, the European Telecommunications Standards Institute (ETSI) sets the respective standards for the usage of these frequencies.
    The main specifications for a single FM broadcasting channels are explained in this section.

    Each channel may allocate a bandwidth of 200 kHz.
    Within a channel, the maximum deviation from the center frequency shall not exceed 75 kHz.
    This leaves a guard band of 25 kHz on either side, to minimize interference with adjacent channels.
    Center frequencies may be allocated on multiples of 100 kHz.

    Fig.\ref{fig_channel_baseband_freqs} shows the upper half-bands' frequency spectrum of a single channel.
    A sum of the left and right audio channel is located starting from 30 Hz, up to 15 kHz.
    The summation of left and right audio channel resembles the mono audio stream.
    The lower limit at 30 Hz is to prevent the transmission of a direct current (DC) part, which would require a large amount of power in transmission. This effect is one of the downsides of amplitude modulation (AM), where the carrier is located at the channels' center frequeny and requires a high transmission power. % TODO: check this fact
    The upper limit at 15 kHz is chosen, to maintain a sufficient spacing to the first subcarrier.
    This subcarrier is allocated at an offset of 19 kHz from the carrier.
    It is also called the 'pilot tone', since it is a continuous signal.
    The pilot tone is used to synchronize the left and right channel of an audio signal, to generate stereo audio.
    The center frequency for stereo audio is located at 38 kHz, which is an integer multiple of 19 kHz for practical reasons.
    However, the 38 kHz carrier is suppressed and thus not visible in the received spectrum.
    It can still be recovered, since it is phase coherent with the 19 kHz pilot tone per definition.
    A signal that is constructed by the difference between left and right audio channel is modulated on a 38 kHz subcarrier.
    The bandwidth for this difference-signal spans 15 kHz on either side of the subcarrier.
    This signal is used to generate a stereo audio signal, in combination with the mono signal.
    This process is explained into more detail in chapter \ref{sec_fm_sig_demod}.
    Considering all these parts in the spectrum, there is still bandwidth free to use, up to the maximum channel bandwitdh 100 kHz in one sideband.
    Because of that, additional services were added to the pure audio transmission, to provide additional data services and information.
    Services that were implemented are the Data Radio Channel (DARC), which is mostly used in Japan and the USA, the Subsidary Communication Authorization (SCA) and the Radio Data System (RDS).
    Out of these, RDS is the most significant service in Europe.
    It is used to transmit additional information about the channel, such as the radio stations' name, currently playing songs' title or traffic information.
    The information about the frequency spectrum was found in \cite{fm_broadcast_tutorial_and_basics}.

    \begin{figure}[!h]
      \centering
        \includegraphics[width=7cm]{img/fm-channel-baseband.png}
      \caption{Allocation of frequencies in an FM channel \cite{ref_fig_channel_freqs}}
      \label{fig_channel_baseband_freqs}
    \end{figure}


%%%%%%%%%%%%%%%%%%%%%%%%%%%%%%%%%%%%%%%%%%%%%%%%%%%%%%%%%%%%%%%%%%%%%%%
\section{Demodulation of a Generic FM Signal}
\label{sec_fm_sig_demod}
  An FM signal that is received by an antenna needs to be demodulated, in order to decode its actual data content.
  The radio-frequency antenna signal usually needs to be amplified, bandpass-filtered, and down-converted to baseband, using subsampling and quadrature-mixing, or similar strategies.
  This part of the signal processing chain is assumed to be working correctly and is out of the scope of this paper.
  The FM signal that is evaluated here, is assumed to be a quadrature-mixed signal in baseband, which means that inphase and quadrature (I/Q-) signals are available.

  %TODO: show downconversion path (maybe as a figure?)
  %https://www.veron.nl/wp-content/uploads/2014/01/FmDemodulator.pdf

  In the following sections, two digital FM demodulator variants are described.
  In general, a FM demodulator serves the purpose to transform FM to AM, so that after the FM demodulator regular AM signal processing techniques can be applied.
  % TODO: add img/discrimination_method_bd2.png here

  \subsection{Frequency Discriminator}
    A frequency discriminator can be defined as a black box component, that generates an output that is directly proportional to the frequency of the input FM signal.
    Different strategies can be implemented in this black box component, the discriminator.
    Here, a differentiator with a subsequent envelope detector is used, which is shown in Fig.\ref{fig_bd_freq_discriminator}.

    \begin{figure}[!h]
      \centering
        \includegraphics[width=7cm]{img/discrimination_method_bd1.jpg}
      \caption{Frequency discriminator block diagram \cite{ref_fig_freq_discriminator}}
      \label{fig_bd_freq_discriminator}
    \end{figure}

    This type of demodulator converts the FM signal into an AM signal.
    However, this signal still contains an FM content.
    The transformation that happens here can be expressed in a formula, by simply differentiating Eq.\ref{equ_fm_func}.
    This operation, with simple algebraic transformations, results in
    \begin{equation}
      \frac{d s(t)}{dt} = A_c 2 \pi \Big[f_c + k_f m(t) \Big] sin \Big(\omega_c t + 2 \pi k_f \int m(t) dt -pi \Big)
    \end{equation}

    A differentiation in hardware is simply performed by substracting two consecutive samples, like
    \begin{equation}
      \frac{d s(t)}{dt} = s(t) - s(t-\Delta t)
      \label{equ_differentiator_two_samples}
    \end{equation}
    where $\Delta t$ is the inverse of the sample frequency.

    It can clearly be seen, that the signal now contains an AM and an FM content.
    The FM part is described by the sine function.
    More importantly the AM part, which resembles the envelope of the signal, is described by the term within the square brackets.
    The resulting signal in time-domain is shown in the top left diagram in Fig.\ref{fig_time_domain_envelope_detect}.
    This signal needs to be fed into an envelope detector, as shown in Fig.\ref{fig_bd_freq_discriminator}.
    It extracts the envelope by removing the high frequency portion.
    The method implemented here is one of the most simple ones and is usually referred to as 'Asynchronous Half-Wave Envelope Detector'\cite{ref_envelope_detector}.
    It consists of a thresholding unit to remove negative values and a conventional lowpass filter.
    In analog implementations, the thresholding unit is a diode.
    The lowpass' cutoff frequency needs to be adapted to the envelope signals' maximum frequency, since it needs to be able to follow the signal, but also sufficiently smooth out the rectified signal.
    Different methods may be implemented for the envelope detector.
    For example, an 'Asynchronous Full-Wave Envelope Detector' can be used.
    Therefore, the mentioned method only needs to swap the thresholding unit with an unit that calculates an absolut value.
    In that architecture, a full-wave rectification is performed on the signal, which significantly improves the envelope detection accuracy \cite{ref_envelope_detector}.
    The envelope is followed more exactly, because of the higher power that is available in the signal, since the entire signal is taken into account, and not only the positive half, as in the previous method.

    \begin{figure}[!h]
      \centering
        \includegraphics[width=8.8cm]{img/envelope-detect-time-domain.jpg}
      \caption{Time-domain signals in envelope detection \cite{ref_roppel}}
      \label{fig_time_domain_envelope_detect}
    \end{figure}

    An important requirement for the differentiator is, that the input signal is of a constant amplitude \cite{ref_schnyder_haller}.
    The transmitted signal is subject to random additive noise on the transmission channel.
    The differentiator would deliver wrong results because of this noise, when substracting two consecutive samples, as described in Eq.\ref{equ_differentiator_two_samples}.
    To achieve this, the complex signal $s(t)$ can be normalized to a value of one.
    \begin{equation}
      s_{norm} = \frac{s}{|s|} = \frac{A(n)\ e^{j\phi_{FM}(n)}}{|{A(n)\ e^{j\phi_{FM}(n)}|}} = e^{j\phi_{FM}(n)} = |1|
    \end{equation}



  \subsection{Phase-Locked Loop}
    A Phase-Locked Loop (PLL) can also be used to transform an FM signal to AM.
    A PLL is a feedback loop that is usually used to generate an output signal that has a fixed phase reference to its input.
    In the case of FM demodulation, the loop filter is configured to be able to follow the frequency variations on the input, which is directly related to the encoded information.
    The output of the PLL directly delivers an AM signal, which corresponds to the transmitted information. \cite{ref_schnyder_haller}

    Due to the limited space in this paper, FM demodulation with a PLL is not described into more detail.


%%%%%%%%%%%%%%%%%%%%%%%%%%%%%%%%%%%%%%%%%%%%%%%%%%%%%%%%%%%%%%%%%%%%%%%
\section{Demodulation of Broadcast FM}
  This chapter is describing the demodulation of the information content of an FM broadcast channel, as it is used in commercial FM radio broadcasting.
  The assumption here is, that the actual FM demodulation, as described in chapter \ref{sec_fm_sig_demod} is already done correctly.
  A frequency spectrum of an FM channel is shown in Fig.\ref{fig_channel_baseband_freqs} above.

  \subsection{Mono}
  \label{demod_mono}
    The first part of the channel frequency spectrum is the summation signal of the left and right audio channel.
    A mono receiver can thus simply replay the so-called multiplex signal, which is generated by the FM demodulation described in chapter \ref{sec_fm_sig_demod}. The 19 kHz pilot tone will not be audible by most people, because it is outside of the range of a human ear. Besides that, any used speaker may also be unable to produce a frequency in this range. This explanation applies to several higher frequency components in the spectrum as well.
    In order to be able to decimate the signal to a lower samplerate, for example to store it in a file with 44.1 kHz samplerate, a lowpass needs to be inserted before the decimation and output.
    The lowpass' cutoff frequency needs to be configured after 15 kHz.
    The filter needs to reach a sufficient attenuation before the pilot tone at 19 kHz.

  \subsection{Stereo}
    Demodulating an FM channel to recover a stereo audio signal requires a more sophisticated approach.
    In the channel frequency spectrum, there are two main parts of the audio signal - the sum and difference of the left and right channel audio signals.
    To combine these to a stereo signal, the following equations can be applied.
    \begin{equation*}
      L = (L+R) + (L-R) = (2)L
    \end{equation*}
    \begin{equation}
      R = (L+R) - (L-R) = (2)R
      \label{equ_stereo_from_sum_diff}
    \end{equation}

    The block diagram in Fig.\ref{fig_bd_stereo_demod} illustrates the signal processing chain for stereo audio. In the block diagram, $x(t)$ represents the multiplex signal, which comes from the FM demodulator. Signals $x_l(t)$ and $x_r(t)$ describe the left and right audio signal, respectively.

    \begin{figure}[!h]
      \centering
        \includegraphics[width=8.8cm]{img/fm-demod-stereo-audio.png}
      \caption{Block diagram of FM stereo audio demodulation \cite{ref_roppel}}
      \label{fig_bd_stereo_demod}
    \end{figure}

    The central branch selects the pilot tone at 19 kHz with a bandpass filter.
    This bandpass needs to have a frequency response that is sharp enough to have a sufficient attenuation at $\pm$4 kHz, since this is where the summation and difference signals frequencies are allocated.
    Afterwards, the pilot tone frequency is doubled via a PLL to achieve a 38 kHz to later demodulate the difference signal phase-coherently.
    The PLL lock indicator can be re-used as an indicator, whether a pilot tone is existent.
    In the upper branch, a bandpass filter selects the difference signal with ranges from 23 to 53 kHz.
    It is then modulated to baseband by the previously generated 38 kHz subcarrier.
    Subsequently, a lowpass limits the signal to the audio signal bandwidth of 15 kHz.
    The lower branch lowpass-filters the summation signal.
    The rightmost part in the diagram performs the combination of summation and difference signals, according to the Eq.\ref{equ_stereo_from_sum_diff}.

%%%%%%%%%%%%%%%%%%%%%%%%%%%%%%%%%%%%%%%%%%%%%%%%%%%%%%%%%%%%%%%%%%%%%%%
\section{Matlab Simulation}
  Matlab is used to simulate an FM demodulator.
  The objective is to demodulate a previously recorded data stream, so that one can listen to a mono audio signal.
  Several steps that are required therefore are described briefly in the following sections.

  % TODO: insert entire block diagram of demodulation
  %       HF => abs() => discriminator => AM demod => audio

  \subsection{Data Acquisition}
    A data stream is recorded, using a cheap RTL-SDR dongle.
    The device can be plugged into USB on a host computer.
    It features an RTL2832U chipset.
    In combination with a host software, this basically assembles a software defined radio (SDR).
    The recorded data stream is stored as a file with interleaved IQ baseband data.

  \subsection{Demodulation Steps}
    The SDR is used to record a data stream with a sampling frequency of 1 MHz, at a center frequency of 98.0 MHz. A dataset of 10 seconds is recorded.
    The aim is to listen to the Austrian radio station OE3, which is broadcasted at a center frequency of 98.1 MHz in the region of Saalfelden.

    \subsubsection{Recorded spectrum}
      The file was recorded at a center frequency $f_c$ of 98 MHz, while the desired station is located at 98.1 MHz.
      This means, that the recorded spectrum is allocated at an intermediate frequency (IF) of 0.1 MHz.

    \subsubsection{Modulate to FM channel center frequency}
      Because the signal needs to be in baseband, to be able to demodulate it, the spectrum needs to be shifted by 0.1 MHz.
      This is achieved by modulating with a complex signal with exactly that frequency.

    \subsubsection{Decimation}
      The signal still has a sample rate of 1 MHz, which is an overkill for the application, since a channels' bandwith only ranges up to 100 kHz.
      Keeping in mind the Nyquist theorem, the sample rate therefore needs to be at least 200 kHz.
      Thus, the signal is decimated by an arbitrary factor of 4 % TODO: more? see Nyquist.. work this out in script
      , which still leaves a rate of x M/kHz.

    \subsubsection{Demodulation}
      Demodulation is done by normalizing the signal amplitude to one.
      A FIR filter is used as a differentiator.
      Envelope detection is omitted for now, for simplicity and because of the mentioned effects in section \ref{demod_mono}.
    \subsubsection{Verification}
      Verification of correct functionality is only done subjectively, by listening to the demodulated audio signal via the PCs' soundcard.
      The OE3 stations' signal can be heard clearly.

\section{Conclusion}
  FM is a well-known technology, that is established in anyones daily life.
  Looking at it from a signal processing perspective is an interesting task, since it requires many principles of digital signal processing, but is still comparably simple to demodulate.

  % An example of a floating figure using the graphicx package.
% Note that \label must occur AFTER (or within) \caption.
% For figures, \caption should occur after the \includegraphics.
% Note that IEEEtran v1.7 and later has special internal code that
% is designed to preserve the operation of \label within \caption
% even when the captionsoff option is in effect. However, because
% of issues like this, it may be the safest practice to put all your
% \label just after \caption rather than within \caption{}.
%
% Reminder: the "draftcls" or "draftclsnofoot", not "draft", class
% option should be used if it is desired that the figures are to be
% displayed while in draft mode.
%
%\begin{figure}[!h]
%  \centering
%    \includegraphics[width=2.5in]{img/fmradio-block-diagram.png}
%  \caption{Block diagram of the entire system.}
%  \label{fig_sim}
%\end{figure}

% Note that the IEEE typically puts floats only at the top, even when this
% results in a large percentage of a column being occupied by floats.


% An example of a double column floating figure using two subfigures.
% (The subfig.sty package must be loaded for this to work.)
% The subfigure \label commands are set within each subfloat command,
% and the \label for the overall figure must come after \caption.
% \hfil is used as a separator to get equal spacing.
% Watch out that the combined width of all the subfigures on a
% line do not exceed the text width or a line break will occur.
%
%\begin{figure*}[!t]
%\centering
%\subfloat[Case I]{\includegraphics[width=2.5in]{box}%
%\label{fig_first_case}}
%\hfil
%\subfloat[Case II]{\includegraphics[width=2.5in]{box}%
%\label{fig_second_case}}
%\caption{Simulation results for the network.}
%\label{fig_sim}
%\end{figure*}
%
% Note that often IEEE papers with subfigures do not employ subfigure
% captions (using the optional argument to \subfloat[]), but instead will
% reference/describe all of them (a), (b), etc., within the main caption.
% Be aware that for subfig.sty to generate the (a), (b), etc., subfigure
% labels, the optional argument to \subfloat must be present. If a
% subcaption is not desired, just leave its contents blank,
% e.g., \subfloat[].


% An example of a floating table. Note that, for IEEE style tables, the
% \caption command should come BEFORE the table and, given that table
% captions serve much like titles, are usually capitalized except for words
% such as a, an, and, as, at, but, by, for, in, nor, of, on, or, the, to
% and up, which are usually not capitalized unless they are the first or
% last word of the caption. Table text will default to \footnotesize as
% the IEEE normally uses this smaller font for tables.
% The \label must come after \caption as always.
%
%\begin{table}[!t]
%% increase table row spacing, adjust to taste
%\renewcommand{\arraystretch}{1.3}
% if using array.sty, it might be a good idea to tweak the value of
% \extrarowheight as needed to properly center the text within the cells
%\caption{An Example of a Table}
%\label{table_example}
%\centering
%% Some packages, such as MDW tools, offer better commands for making tables
%% than the plain LaTeX2e tabular which is used here.
%\begin{tabular}{|c||c|}
%\hline
%One & Two\\
%\hline
%Three & Four\\
%\hline
%\end{tabular}
%\end{table}


% Note that the IEEE does not put floats in the very first column
% - or typically anywhere on the first page for that matter. Also,
% in-text middle ("here") positioning is typically not used, but it
% is allowed and encouraged for Computer Society conferences (but
% not Computer Society journals). Most IEEE journals/conferences use
% top floats exclusively.
% Note that, LaTeX2e, unlike IEEE journals/conferences, places
% footnotes above bottom floats. This can be corrected via the
% \fnbelowfloat command of the stfloats package.



% conference papers do not normally have an appendix


% use section* for acknowledgment
% \section*{Acknowledgment}


% The authors would like to thank...





% trigger a \newpage just before the given reference
% number - used to balance the columns on the last page
% adjust value as needed - may need to be readjusted if
% the document is modified later
%\IEEEtriggeratref{8}
% The "triggered" command can be changed if desired:
%\IEEEtriggercmd{\enlargethispage{-5in}}

% references section

% can use a bibliography generated by BibTeX as a .bbl file
% BibTeX documentation can be easily obtained at:
% http://mirror.ctan.org/biblio/bibtex/contrib/doc/
% The IEEEtran BibTeX style support page is at:
% http://www.michaelshell.org/tex/ieeetran/bibtex/
%\bibliographystyle{IEEEtran}
% argument is your BibTeX string definitions and bibliography database(s)
%\bibliography{IEEEabrv,../bib/paper}
%
% <OR> manually copy in the resultant .bbl file
% set second argument of \begin to the number of references
% (used to reserve space for the reference number labels box)
\begin{thebibliography}{1}
  \bibitem{itu_regulations}
    International Telecommunication Union (ITU), \emph{Radio Regulation Articles, Edition of 2020}, Vol. I-EA5

  \bibitem{ebu_fm_modulators_in_europe_power_regulations}
    European Broadcasting Union (EBU), \emph{Low-power FM modulators in Europe}, EBU-R120-2007, \url{https://tech.ebu.ch/docs/r/r120.pdf}, pg.5

  \bibitem{fm_broadcast_tutorial_and_basics}
    Electronics Notes, \emph{Broadcast VHF FM Tutorial \& Basics}, \url{https://www.electronics-notes.com/articles/audio-video/broadcast-audio/vhf-fm-frequency-modulation-basics.php}, opened Dec. 10th, 2020

  \bibitem{FM_Maths_Info_1}
    \url{https://fas.org/man/dod-101/navy/docs/es310/FM.htm}, opened on Dec. 12th, 2020

  \bibitem{FMMaths2}
    \url{https://www.tutorialspoint.com/analog_communication/analog_communication_angle_modulation.htm}, opened on Dec. 12th, 2020

  \bibitem{ref_fig_channel_freqs}
    Figure: Baseband Signal of an FM channel, \url{https://de.wikipedia.org/wiki/UKW-Rundfunk#Technische_Details}, opened on Dec. 11th, 2020

  \bibitem{ref_fig_freq_discriminator}
    Figure: Frequency Discriminator Block Diagram, \url{https://www.tutorialspoint.com/analog_communication/analog_communication_fm_demodulators.htm}, opened on Dec. 16th, 2020

  \bibitem{ref_envelope_detector}
    Rick Lyons, \emph{Digital Envelope Detection: The Good, the Bad, and the Ugly}, April 3, 2016, \url{https://www.dsprelated.com/showarticle/938.php}, opened on Dec. 16th, 2020

  \bibitem{ref_roppel}
    C.Roppel, \emph{Analoge Modulationsverfahren und Rundfunktechnik} Begleitmaterial zum Buch 'Grundlagen der digitalen Kommunikationstechnik, Übertragungstechnik - Signalverarbeitung - Netze', \url{https://www.hs-schmalkalden.de/fileadmin/portal/Dokumente/Fakult%C3%A4t_ET/Personal/Roppel/Buch/Analoge_Modulationsverfahren.pdf}, opened on Dec. 16th, 2020

  \bibitem{ref_schnyder_haller}
    F.Schnyder, C.Haller, \emph{Implementation of FM Demodulator
    Algorithms on a High Performance Digital Signal Processor}, 2002, \url{https://www.veron.nl/wp-content/uploads/2014/01/FmDemodulator.pdf}
\end{thebibliography}




% that's all folks
\end{document}


